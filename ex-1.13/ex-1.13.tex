\documentclass[11pt, oneside]{article}
\usepackage{geometry}
\geometry{letterpaper}
\usepackage{graphicx}
\usepackage{amssymb}
\usepackage{amsmath}
\usepackage{ntheorem}

\title{SICP Exercise 1.13}
\author{Yuchong Pan}

\begin{document}
\maketitle

\theoremstyle{nonumberplain}

\newtheorem{claim}{Claim}
\newtheorem{lemma}{Lemma}
\newtheorem{proof}{Proof}

\begin{lemma}
    $Fib(n) = \frac{\phi^n-\psi^n}{\sqrt{5}}$, where $\phi = \frac{1+\sqrt{5}}{2}$ and $\psi = \frac{1-\sqrt{5}}{2}$.
\end{lemma}

\begin{proof}
    We have the following degenerate cases:
    $$\frac{\phi^0-\psi^0}{\sqrt{5}} = \frac{1-1}{\sqrt{5}} = 0 = Fib(0)$$
    $$\frac{\phi^1-\psi^1}{\sqrt{5}} = \frac{\sqrt{5}}{\sqrt{5}} = 1 = Fib(1)$$

    Suppose we have proved that
    $$Fib(i) = \frac{\phi^i-\psi^i}{\sqrt{5}}$$
    $$Fib(i+1) = \frac{\phi^{i+1}-\psi^{i+1}}{\sqrt{5}}$$

    According to the definition of the Fibonacci numbers, we have
    $$Fib(i+2) = Fib(i)+Fib(i+1) = \frac{\left(\phi^i+\phi^{i+1}\right)-\left(\psi^i+\psi^{i+1}\right)}{\sqrt{5}}$$

    Since we have $\phi^2 = \phi+1$ and $\psi^2 = \psi+1$, then
    $$\phi^{i+2} = \phi^i\phi^2=\phi^i(\phi+1)=\phi^{i+1}+\phi^i$$
    $$\psi^{i+2} = \psi^i\psi^2=\psi^i(\psi+1)=\psi^{i+1}+\psi^i$$

    Hence, we have
    $$Fib(i+2) = \frac{\phi^{i+2} - \psi^{i+2}}{\sqrt{5}}$$.

    By induction, it can be proved that
    $$Fib(n) = \frac{\phi^n-\psi^n}{\sqrt{5}}$$
\end{proof}

\begin{claim}
    $Fib(n)$ is the closest integer to $\frac{\phi^n}{\sqrt{5}}$, where $\phi = \frac{1+\sqrt{5}}{2}$.
\end{claim}

\begin{proof}
    Since $0 < \frac{1}{\sqrt{5}} < \frac{1}{2}$ and since $-\frac{1}{2} < \frac{\psi}{\sqrt{5}} < 0$, then we have
    $$-\frac{1}{2} < \frac{\psi^n}{\sqrt{5}} < \frac{1}{2}, n \in \mathbb{N}$$

    By the lemma we have proved that
    $$Fib(n) = \frac{\phi^n-\psi^n}{\sqrt{5}}$$

    Hence, $Fib(n)$ is the closest integer to $\frac{\phi^n}{\sqrt{5}}$.
\end{proof}

\end{document}
