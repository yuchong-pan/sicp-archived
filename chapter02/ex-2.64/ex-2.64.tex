\documentclass[11pt, oneside]{article}
\usepackage{geometry}
\geometry{letterpaper}
\usepackage{graphicx}
\usepackage{amssymb}
\usepackage{amsmath}
\usepackage{tikz}
\usepackage{tikz-qtree}
\usepackage{url}

\title{SICP Exercise 2.64}
\author{Yuchong Pan}

\begin{document}
\maketitle

a. If the parameter $n$ equals 0, then the constructed tree will be empty, and the remaining elements will be all the original elements. Otherwise, the \url{partial-tree} procedure recursively converts the first $\lfloor \frac{n-1}{2} \rfloor$ elements to the left subtree and the last $n - \left( \lfloor \frac{n-1}{2} \rfloor + 1 \right)$ elements to the right subtree, and the middle entry is the entry of the constructed tree. Then, \url{partial-tree} calls the constructor \url{make-tree} to make a tree with the entry, the left subtree and the right subtree. The tree produced by \url{list->tree} for the list (1 3 5 7 9 11) is given as follows:

~

~

~

\begin{center}
    \begin{tikzpicture}
        \node (isroot) {5}
            [sibling distance=5cm]
            child {
                node {1}
                    [sibling distance=2.5cm]
                    child[missing]
                    child { node {3} }
            }
            child {
                node {9}
                    [sibling distance=2.5cm]
                    child { node {7} }
                    child { node {11} }
            };
    \end{tikzpicture}
\end{center}

~

~

~

b. The number of steps required by \url{list->tree} grows as $\Theta(n)$, where $n$ is the size of the given list.

\end{document}
