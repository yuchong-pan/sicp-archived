\documentclass[11pt, oneside]{article}
\usepackage{geometry}
\geometry{letterpaper}
\usepackage{graphicx}
\usepackage{amssymb}
\usepackage{amsmath}
\usepackage{tikz}
\usepackage{tikz-qtree}
\usepackage{url}

\title{SICP Exercise 2.64}
\author{Yuchong Pan}

\begin{document}
\maketitle

The tree for $n = 5$ is given as follows:

\begin{center}
    \begin{tikzpicture}
        \node (isroot) {31}
            [sibling distance=2.5cm]
            child {
                node {15}
                    [sibling distance=2.5cm]
                    child {
                        node {7}
                            [sibling distance=2.5cm]
                            child {
                                node {3}
                                    [sibling distance=2.5cm]
                                    child { node {1} }
                                    child { node {2} }
                            }
                            child { node {4} }
                    }
                    child { node {8} }
            }
            child { node {16} };
    \end{tikzpicture}
\end{center}

The tree for $n = 10$ is given as follows:

\begin{center}
    \begin{tikzpicture}
        \node (isroot) {1023}
            [sibling distance=2.5cm]
            child {
                node {511}
                    [sibling distance=2.5cm]
                    child {
                        node {255}
                            [sibling distance=2.5cm]
                            child {
                                node {127}
                                    [sibling distance=2.5cm]
                                    child {
                                        node {63}
                                            [sibling distance=2.5cm]
                                            child {
                                                node {31}
                                                    [sibling distance=2.5cm]
                                                    child {
                                                        node {15}
                                                            [sibling distance=2.5cm]
                                                            child {
                                                                node {7}
                                                                    [sibling distance=2.5cm]
                                                                    child {
                                                                        node {3}
                                                                            [sibling distance=2.5cm]
                                                                            child { node {1} }
                                                                            child { node {2} }
                                                                    }
                                                                    child { node {4} }
                                                            }
                                                            child { node {8} }
                                                    }
                                                    child { node {16} }
                                            }
                                            child { node {32} }
                                    }
                                    child { node {64} }
                            }
                            child { node {128} }
                    }
                    child { node {256} }
            }
            child { node {512} };
    \end{tikzpicture}
\end{center}

In such a tree (for general $n$), one bit is required to encode the most frequent symbol, and $(n-1)$ bits are required to encode the least frequent symbol.

\end{document}
